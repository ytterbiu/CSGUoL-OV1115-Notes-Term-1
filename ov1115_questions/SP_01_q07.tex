%% Specimen Paper 1 %%
% Question 7
\question The time period, $T$, of a simple pendulum of length, $L$, 
is related to the acceleration of free fall, $g$, by the formula:
\[
	T = 2\pi \sqrt{\frac{L}{g}}
\]
\begin{parts}
	\part Calculate $T$ when $L=10$ and $\pi=3.14$ and $g=9.8$. 
		(Your answer should be to 1 decimal place but does not require units)
	\part Rearrange the formula to make $L$ the subject.
\end{parts}
\begin{solution}
	\begin{parts}
		\part %part a
			\[
				T = 2\pi \sqrt{\frac{L}{g}}
			\]
			\[
				\rightarrow
				T = 2(3.14) \sqrt{\frac{10}{9.8}}
			\]
			\[
				T = 6.34375797\ldots
			\]
			\qSolMath{
				T = 6.3
			}{
				(1dp)
			}
		\part %part b
			\noindent
			\begin{align*}
				T 
					&= 2\pi \sqrt{\frac{L}{g}}
					\\
				\sqrt{\frac{L}{g}} 
					&= \frac{T}{2\pi}
					\\
				\frac{L}{g} 
					&= \left( \frac{T}{2\pi} \right)^{2}
			\end{align*}
			\qSolMath{
				L = g \left( \frac{T}{2\pi} \right)^{2}
			}{}
			or
			\[
				L = \frac{g T^{2}}{4\pi^{2}}
			\]
	\end{parts}
\end{solution}

\appenddata{questionSolutions}{
{
\begin{parts}
	\part $T = 6.3$ (1dp)
	\part $L = g \left( \dfrac{T}{2\pi} \right)^{2}$ 
		\newline 
		or 
		\newline
		$L = \dfrac{g T^{2}}{4\pi^{2}}$
\end{parts}
}
}